
\chapter{Windows7とubuntu15.10のデュアルブート}
純粋にプログラミングをするだけであれば、Windowsやcygwinは
煩わしい要素が多い。CUDAを利用する場合も基本的には
Visual Studio Communityのcl.exeに縛られることになる。
そこでLinux/ubuntuをデュアルブートで利用することで
快適なコーディング及び実行環境を用意することは意味がある。
ただ環境整備に失敗するとWindowsが破壊されて起動すら
できなくなるので手順等についてここにまとめる。
\section{PC起動時のフロー}
インストール前にPC起動時に何が起こっているか
知っておくことが非常に重要である。
PCの電源をつけるとまずBIOSが立ち上がり
起動ドライブを探す。
起動ドライブが決まったらそのドライブの先頭512 byteを読み込む。
この領域にはMBR (Master Boot Recode)と
呼ばれるものが置かれている。
MBRはブートローダ (466 byte)
・パーティションテーブル ( 64 byte)
・ブートシグニチャ (2 byte)で構成され、
ブートローダはOSを起動するための別ブートローダを呼び出す。
ドライブ先頭に置かれているブートローダを始点として多段階的に
ブートローダが呼び出されていって
最終的にOSが立ち上がることになる。

同じドライブ内で
パーティションをわけてデュアルブートを導入すると、
MBRが書き換えられたり新しいブートローダを経由することになる。
インストール及びアンインストールの手順を誤ると
ブートローダが読み込めなくなり起動ができなくなる。
この状態に陥るとOSインストールディスクを
利用してMBRを元の状態に修復しなくてはならなくなるので
初心者には相当厳しい。

このようなリスクからデュアルブートで追加する新規OSは
パーティションを分けるのではなく別ドライブに入れるのが良い。
そうすれば元々の既存ドライブのMBRは書き込まれることはないので
起動不可になる最悪の事態は簡単に避けられる。
MBRの書き換えだけでなく個人データの消去リスクなども考えて、
新規OSを入れる際は既存ドライブをPCから取り外して
作業を進めると良いだろう。

\section{ubuntuのインストール}
初めにダウンロードサイト\cite{ubuntu_download}から
unbuntuのインストールに必要となる.isoをダウンロードする。
64bit日本語環境である\texttt{ubuntu-ja-15.10-desktop-amd64.iso}を今回利用した。
.isoはそのままではインストールに利用できないので
USBWriter\cite{USBWriter}というソフトを利用してUSBにインストール実行データとして書き込む。
USBの中身は全削除されるので注意が必要だ。

次にubuntuを入れることになる新規ドライブのフォーマットを行う。
Windows7であればスタートメニューのコンピュータを右クリックして管理を選択。
立ち上がるウィンドウからディスクを管理を選択。
フォーマットする新規ドライブを右クリックして新しいシンプルボリュームの作成で
FAT32形式を指定してフォーマットを行う。
この作業は本来必要ないこともあるが自分の環境ではこれで上手く行ったので記しておいた。

次に一度PCの電源を落として、Windowsの起動ドライブとなっているデバイスをPCから取り外す(前述した保険)。
先ほど用意したインストール用のUSBを挿してPCを起動する。
BIOS画面が立ち上がったらF12を押すことで起動するドライブを
選択できるようになるのでUSBを選択する。
ubuntuのインストール画面が立ち上がったら日本語を選択して次の画面へ。
インストール中にアップデートをダウンロードする、
サードパーティのソフトウェアをインストールするにチェックを入れて次の画面へ。
「インストールの種類」の画面で''それ以外''にチェックを入れて次の画面へ(※重要)。
次にパーティションについて設定を行う。今回使ったドライブは256 GBのSSDであったので、
ubuntuの領域として100 GB・利用方法をext4・マウントポイント/のパーティションを作成、
swap領域として4 GB・利用方法swapのパーティションを作成、
WindowsからSSDの保存領域を利用したかったので
150 GB・利用方法FAT32・マウントポイント/windowsのパーティションを作成した。
またブートローダをインストールするデバイスはubuntuの領域として用意したパーティションを選択して次の画面へ。
PCの名前やユーザーIDやパスワードを選択してインストールを開始する。
インストールが完了したらダイアログにしたがって再起動を行う。

上手く行っていればubuntuが立ち上がる。ubuntuの起動に成功したら、
再度PCの電源を落として既存ドライブを接続する。
BIOS画面でF12を押して、Windowsの入っているドライブを
選択すればWindowsが立ち上がり、
ubuntuが入っているドライブを選択すればWindowsが立ち上がることを確認する。
Windows側から保存領域として確保したSSDの150 GBが認識できるかも確かめる。

\section{ubuntuの初期設定}
メニューのゲストセッションの項目が不要なので削除する。
/usr/share/lightdm/lightdm.conf.dフォルダ内にある
LightDMの設定ファイル50-ubuntu.confを編集する。\\
\texttt{\$ sudo gedit /usr/share/lightdm/lightdm.conf.d/50-ubuntu.conf}\\
テキストの最後に\texttt{allow-guest=false}を追加する。

ドライバなどを更新する。
ここではビデオカードのドライバ更新を例に上げる。\\
\texttt{\$ sudo apt-get update}\\
\texttt{\$ sudo apt-cache search `nvidia-[0-9]+\$'}\\
中から最新版と思われるものをインストールする。\\
\texttt{\$ sudo apt-get install nvidia-352}
apt-getが基本万能なので利用する。
これを利用してemacs,git,tig,tmuxなどは簡単に導入できる。

新しいパッケージをインストールする場合は
debファイルをwgetコマンドなどを利用してダウンロードして
dpkgコマンドからインストールする。
CUDA Toolkitを例に挙げると以下のようになる。\\
\texttt{\$ sudo dpkg -i cuda-repo-ubuntu1504-7-5-local\_7.5-18\_amd64.deb}\\
\texttt{\$ sudo apt-get update}\\
\texttt{\$ sudo apt-get install cuda}\\
余談だが、ubuntuでCUDAを利用するためには環境変数を追加する必要がある。
\texttt{.bachrc}などに以下の記述をしておけばよいだろう。\\
\texttt{\$ export PATH=/usr/local/cuda-7.5/bin:\$PATH}\\
\texttt{\$ export LD\_LIBRARY\_PATH=/usr/local/cuda-7.5/lib64:\$LD\_LIBRARY\_PATH}\\
既に入っているパッケージ等の確認は以下のように行う。\\
\texttt{\$ sudo dpkg -l | grep ***}



























