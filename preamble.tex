%\documentclass[a4paper,11pt]{jarticle}
%\documentclass[a4paper,12pt]{jsbook}
\documentclass[a4paper,11pt,onecolumn,twoside,openright]{jsbook}


%
%  preamples 
%


% ** User Preference Options **
\usepackage{sty/thesis}

%% find old command or package
%\RequirePackage[l2tabu, orthodox]{nag} 

%% graphics
\usepackage[dvipdfmx]{graphicx}
\graphicspath{{fig/}}

%% subfigure
\usepackage{subcaption}

%% color
\usepackage[dvipdfmx]{color}

%% yohaku
\usepackage[driver=dvipdfm,truedimen,margin=2cm]{geometry}
\geometry{left=30mm,right=30mm,top=35mm,bottom=30mm} % for yohaku
\usepackage{here}

%% math
%\usepackage[all, warning]{onlyamsmath} % warn if using mathematics that amsmath don't provide
\usepackage{amsmath}
\usepackage{mathptmx}
\usepackage{amssymb}

%% fonts
%\usepackage{txfonts}
%\usepackage{newtxtext,newtxmath}
\usepackage{mathptmx}
\usepackage{bm}
\usepackage{amssymb}

% subsection display
\setcounter{tocdepth}{2}

%% algorithm
%\usepackage{algpseudocode,algorithm}

%% table
\usepackage{multirow}

%% hyper link
\usepackage[
    dvipdfmx,
    linktocpage=true,
    bookmarkstype=toc=true,
    colorlinks=true,
    allcolors=black
]{hyperref}
\usepackage{pxjahyper}

% for warning
\input{sty/jdummy.def}

%
% macros
%

% 画像引用
\newcommand{\Figref}[1]{\figurename~\ref{#1}}
% 表引用
\newcommand{\Tabref}[1]{\tablename~\ref{#1}}
% 数式引用
\newcommand{\Eqref}[1]{式~(\ref{#1})}

% 微分
\newcommand{\diff}[2]{\cfrac{\mathrm{d}#1}{\mathrm{d} #2}}
\newcommand{\cdiff}[2]{\cfrac{\mathrm{d}#1}{\mathrm{d} #2}}
\newcommand{\tdiff}[2]{\tfrac{\mathrm{d}#1}{\mathrm{d} #2}}
\newcommand{\ddiff}[2]{\frac{\mathrm{d}^2#1}{\mathrm{d} #2^2}}
\newcommand{\cddiff}[2]{\cfrac{\mathrm{d}^2#1}{\mathrm{d} #2^2}}
\newcommand{\tddiff}[2]{\tfrac{\mathrm{d}^2#1}{\mathrm{d} #2^2}}
% 偏微分
\newcommand{\pdiff}[2]{\cfrac{\partial #1}{\partial #2}}
\newcommand{\cpdiff}[2]{\cfrac{\partial #1}{\partial #2}}
\newcommand{\tpdiff}[2]{\tfrac{\partial #1}{\partial #2}}
\newcommand{\pddiff}[2]{\frac{\partial^2 #1}{\partial #2^2}}
\newcommand{\cpddiff}[2]{\cfrac{\partial^2 #1}{\partial #2^2}}
\newcommand{\tpddiff}[2]{\tfrac{\partial^2 #1}{\partial #2^2}}

% 逆数
\newcommand{\rev}[1]{\frac{1}{#1}}
\newcommand{\crev}[1]{\cfrac{1}{#1}}
\newcommand{\trev}[1]{\tfrac{1}{#1}}

% 斜体Δ
\newcommand{\D}{\varDelta}
\newcommand{\Dt}{\varDelta t}
\newcommand{\Dx}{\varDelta x}
\newcommand{\Dy}{\varDelta y}
\newcommand{\Dz}{\varDelta z}

% ∇
\newcommand{\Grad}{\nabla}
\newcommand{\Div}{\nabla\cdot}
\newcommand{\Rot}{\nabla\times}

\newcommand{\kagi}[1]{\textless#1\textgreater} % angle brackets < >

% 括弧
\newcommand{\kakko}[1]{\left(#1\right)} % parenthesis ( )
\newcommand{\ckakko}[1]{\left\{#1\right\}} % curly braces { }
\newcommand{\skakko}[1]{\left[#1\right]} % square blackets [ ]
\newcommand{\akakko}[1]{\left\langle#1\right\rangle} % angle brackets < >
\newcommand{\abs}[1]{\left|#1\right|} % absolute value | |
\newcommand{\norm}[1]{\left\|#1\right\|} % norm || ||
\newcommand{\floor}[1]{\left\lfloor#1\right\rfloor} % floor

% 2列ベクトル
\newcommand{\vectwo}[2]{\begin{bmatrix}#1\\#2\end{bmatrix}}
\newcommand{\vecthree}[3]{\begin{bmatrix}#1\\#2\\#3\end{bmatrix}}
\newcommand{\vecfive}[5]{\begin{bmatrix}#1\\#2\\#3\\#4\\#5\end{bmatrix}}
\newcommand{\vecsix}[6]{\begin{bmatrix}#1\\#2\\#3\\#4\\#5\\#6\end{bmatrix}}

% pmatrix, vmatrix のショートカット
\newcommand{\pmat}[1]{\begin{pmatrix}#1\end{pmatrix}}
\newcommand{\vmat}[1]{\begin{vmatrix}#1\end{vmatrix}}

% 場合分け
\newcommand{\case}[1]{\begin{cases}#1\end{cases}}

% displaystyle
\newcommand{\disp}{\displaystyle}

% for RK pth step
\newcommand{\step}[1]{{\langle #1 \rangle}}

% ベクトル用太字の定義
\bmdefine{\Va}{\bm{a}}
\bmdefine{\Vb}{\bm{b}}
\bmdefine{\Vc}{\bm{c}}
\bmdefine{\Vd}{\bm{d}}
\bmdefine{\Ve}{\bm{e}}
\bmdefine{\Vf}{\bm{f}}
\bmdefine{\Vg}{\bm{g}}
\bmdefine{\Vh}{\bm{h}}
\bmdefine{\Vi}{\bm{i}}
\bmdefine{\Vj}{\bm{j}}
\bmdefine{\Vk}{\bm{k}}
\bmdefine{\Vl}{\bm{l}}
\bmdefine{\Vm}{\bm{m}}
\bmdefine{\Vn}{\bm{n}}
\bmdefine{\Vo}{\bm{o}}
\bmdefine{\Vp}{\bm{p}}
\bmdefine{\Vq}{\bm{q}}
\bmdefine{\Vr}{\bm{r}}
\bmdefine{\Vs}{\bm{s}}
\bmdefine{\Vt}{\bm{t}}
\bmdefine{\Vu}{\bm{u}}
\bmdefine{\Vv}{\bm{v}}
\bmdefine{\Vw}{\bm{w}}
\bmdefine{\Vx}{\bm{x}}
\bmdefine{\Vy}{\bm{y}}
\bmdefine{\Vz}{\bm{z}}

\bmdefine{\VA}{\bm{A}}
\bmdefine{\VB}{\bm{B}}
\bmdefine{\VC}{\bm{C}}
\bmdefine{\VD}{\bm{D}}
\bmdefine{\VE}{\bm{E}}
\bmdefine{\VF}{\bm{F}}
\bmdefine{\VG}{\bm{G}}
\bmdefine{\VH}{\bm{H}}
\bmdefine{\VI}{\bm{I}}
\bmdefine{\VJ}{\bm{J}}
\bmdefine{\VK}{\bm{K}}
\bmdefine{\VL}{\bm{L}}
\bmdefine{\VM}{\bm{M}}
\bmdefine{\VN}{\bm{N}}
\bmdefine{\VO}{\bm{O}}
\bmdefine{\VP}{\bm{P}}
\bmdefine{\VQ}{\bm{Q}}
\bmdefine{\VR}{\bm{R}}
\bmdefine{\VS}{\bm{S}}
\bmdefine{\VT}{\bm{T}}
\bmdefine{\VU}{\bm{U}}
\bmdefine{\VV}{\bm{V}}
\bmdefine{\VW}{\bm{W}}
\bmdefine{\VX}{\bm{X}}
\bmdefine{\VY}{\bm{Y}}
\bmdefine{\VZ}{\bm{Z}}

\bmdefine{\Vzero}{\bm{0}}

\bmdefine{\Vphi}{\bm{\phi}}
\bmdefine{\Vsigma}{\bm{\sigma}}
\bmdefine{\Vtheta}{\bm{\theta}}
\bmdefine{\VTheta}{\bm{\Theta}}
%%%%%%%%%%%%%%%%%%%%%%%%%%%%%%%%
